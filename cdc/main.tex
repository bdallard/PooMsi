\documentclass[12pt]{article}
\usepackage[english]{babel}
\usepackage[utf8x]{inputenc}
\usepackage{amsmath}
\usepackage{graphicx}
\usepackage[colorinlistoftodos]{todonotes}
\usepackage{fancyhdr}
\usepackage{caption}

\begin{document}

\begin{titlepage}

\newcommand{\HRule}{\rule{\linewidth}{0.5mm}} % Defines a new command for the horizontal lines, change thickness here

\center % Center everything on the page
 
%----------------------------------------------------------------------------------------
%	HEADING SECTIONS
%----------------------------------------------------------------------------------------
\title{Next Customer Ventures}

\textsc{\LARGE IESA}\\[1.8cm] %TITRE 
\textsc{\Large Benjamin Dallard}\\[0.5cm] % AUTEUR


%----------------------------------------------------------------------------------------
%	TITLE SECTION
%----------------------------------------------------------------------------------------

\HRule \\[0.4cm]
{ \huge \bfseries Cahier des charges template}\\[0.4cm] % Title of your document
\HRule \\[1.5cm]
 
%----------------------------------------------------------------------------------------
%	AUTHOR SECTION
%----------------------------------------------------------------------------------------



% If you don't want a supervisor, uncomment the two lines below and remove the section above
%\Large \emph{Author:}\\
%John \textsc{Smith}\\[3cm] % Your name

%----------------------------------------------------------------------------------------
%	DATE SECTION
%----------------------------------------------------------------------------------------

{Janvier 2019}\\[2cm] % Date, change the \today to a set date if you want to be precise

%----------------------------------------------------------------------------------------
%	LOGO SECTION
%----------------------------------------------------------------------------------------

\includegraphics [width=0.6\textwidth] {IESA_logo.jpg} % Include a department/university logo - this will require the graphicx package

%\includegraphics[width=0.5\textwidth]{my-uploaded-figure.png}

%----------------------------------------------------------------------------------------

\vfill % Fill the rest of the page with whitespace

\end{titlepage}

\tableofcontents{}
\newpage

\section{Fondement}

\subsection{Contexte}

\subsection{Enjeux}

\subsection{Les objectifs}
\subsubsection{Les objectifs principaux}


\subsubsection{Les objectifs secondaires}

\newpage
\section{Personnes et organismes impliqués dans les enjeux du projet}

\subsection{Maitrise d'ouvrage}

\subsection{Les développeurs}


\newpage
\section{Utilisateurs du produit}
\subsection{Utilisateurs directs du produit}

\subsection{Implication nécéssaire de la part des utilisateurs dans le projet}

\subsection{Utilisateurs concernés par les opération de maintenance du produit}


\newpage
\section{Les contraintes}

\subsection{Les contraintes non négociables}

\subsubsection{Contraintes sur la conception de la solution}

\subsubsection{Applications partenaires avec lesquelles le produit doit collaborer}

\subsubsection{Lieux de fonctionnement prévus}
\subsubsection{Temps de développement}
\subsubsection{Budget affecté au projet}


\newpage
\section{Exigences fonctionnelles}
\subsubsection{La situation actuelle : état des lieux de la solution ShopinZen}

\subsection{Les documents de spécification}
\subsubsection{Les cas d'utilisations}

\subsubsection{Les diagrammes d'activité}

\subsubsection{Diagramme de classes}
\subsection{L'architecture du projet}
\subsubsection{Architecture cloud}
\subsubsection{Architecture client}
\subsubsection{Déploiement \& JOBS} 

\newpage
\section{Exigences non fonctionnelles}
\subsection{Ergonomie et convivialité du produit}
\subsubsection{Le style du produit packaging inclus}

\subsection{Facilité d'utilisation et facteurs humains}
\subsubsection{Facilité d'utilisation}
\subsubsection{Personnalisation et internalisation}
\subsubsection{Facilité de compréhension}


\subsection{Fonctionnement du produit}
\subsubsection{Rapidité d'execution et temps de latence}
\subsubsection{Capacité de stockage et montée en charge}


\end{document}



